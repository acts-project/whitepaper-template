\documentclass[12pt,a4paper]{scrarticle}

\usepackage{theme/acts}

\usepackage{lipsum}

\title{A whitepaper about a topic}
\author{First Author \and Second Author}

%% Only use basic LaTeX markup here, it gets rendered by MathJax.
\iffalse
%% BEGIN SHORT DESCRIPTION
	This is a whitepaper example. It contains a number of example
	patterns, layouts etc.
%% END SHORT DESCRIPTION
\fi


\begin{document}

\maketitle

\begin{abstract}
    \lipsum[2-4] 
\end{abstract}

\tableofcontents

\section{Introduction}

We closely follow the description in \cite{adams1995}.

\begin{figure}[ht]
    \centering
    \includegraphics[width=0.7\textwidth]{example-image-a}
    \caption{This is a caption for a figure!}
    \label{fig:example_a}
\end{figure}

\lipsum[1] 

As discussed in \autoref{fig:example_a}, the results are very promising.

\section{Method}

\begin{figure}[ht]
    \begin{subfigure}{0.5\textwidth}
        \centering
        \includegraphics[width=\textwidth]{example-image-a}
        \caption{This is a caption for a figure!}
        \label{fig:sub:example_a}
    \end{subfigure}
    \begin{subfigure}{0.5\textwidth}
        \centering
        \includegraphics[width=\textwidth]{example-image-b}
        \caption{This is a caption for a figure!}
        \label{fig:sub:example_b}
    \end{subfigure}

    \caption{This is a combined caption, \subref{fig:sub:example_a} shows one
    thing, \subref{fig:sub:example_b} shows another thing}
    \label{fig:sub}
\end{figure}

\lipsum[1] 

See \autoref{fig:sub} for an example of a combined figure.

\lipsum[1] 

\begin{equation}
    a^2 + b^2 = c^2
    \label{eq:pythagoras}
\end{equation}

The business with the triangles, our boy Pythagoras has already figured out, according to \autoref{eq:pythagoras}.

\section{Conclusion}

\lipsum[1] 

\appendix

\section{Auxiliary material}

\lipsum[1] 

\printbibliography{}

\end{document}
